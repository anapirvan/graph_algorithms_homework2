\subsection*{a)}

Let $T$ be the set of edges of a minimum-cost partial tree of the graph $G$. Since $c_1 < c_2$, according to the greedy principle (specific to Kruskal’s algorithm), the construction of the tree $T$ prioritizes including low-cost edges. Thus, $T$ must include a maximum number of edges from $E_1$ that do not form cycles.

The edges selected from $E_1$ generate an acyclic partial subgraph that preserves the same connected components as the graph $G_1$. Let $p_1$ be the number of these connected components. We know the property that an acyclic graph with $n$ vertices and $p_1$ connected components has exactly $n - p_1$ edges. Therefore, the number of edges of cost $c_1$ in the tree $T$ is $n - p_1$.

For $T$ to become a partial tree of the entire graph $G$, the $p_1$ disjoint connected components must be interconnected. This requires adding exactly $p_1 - 1$ additional edges. Since the potential edges of cost $c_1$ have been exhausted (any other edge from $E_1$ would create a cycle), the remaining connections must necessarily be selected from $E_2$.

In conclusion, the total cost of the partial tree is the sum of the costs of the two categories of edges:
\[ \text{Costul minim al lui T} = c_1 \cdot (n - p_1) + c_2 \cdot (p_1 - 1). \]

\subsection*{b)}

The standard Kruskal algorithm has a complexity of $\mathcal{O}(|E| \log |V|)$, given by the edge-sorting step.
In our case, since the cost function has only two values ($c_1$ and $c_2$), we can completely eliminate the sorting step, reducing the complexity to linear time.

\textbf{Algorithm Description:}

\begin{enumerate}
    \item \textbf{Initialization:} We create a Union–Find structure for the $n$ nodes. Initially, each node is an isolated component.
    
    \item \textbf{Processing edges of minimum cost ($E_1$):}
    Instead of sorting, we directly traverse the list of edges with cost $c_1$. For each edge $(u, v) \in E_1$:
    \begin{itemize}
        \item If $\text{Find}(u) \neq \text{Find}(v)$ (they do not form a cycle), we merge the components ($\text{Union}(u, v)$) and add the edge to the solution.
    \end{itemize}
    
    \item \textbf{Processing edges of maximum cost ($E_2$):}
    After exhausting all edges of cost $c_1$, we traverse the list of edges with cost $c_2$. For each edge $(u, v) \in E_2$:
    \begin{itemize}
        \item If $\text{Find}(u) \neq \text{Find}(v)$, we merge the components and add the edge to the solution.
        \item We may stop once we have selected a total of $n-1$ edges.
    \end{itemize}
\end{enumerate}

\textbf{Complexity Analysis:}
\begin{itemize}
    \item In the absence of the sorting operation, the running time is determined solely by the calls made to the \textit{Union–Find} structure.
    \item In conclusion, the algorithm effectively runs in \textbf{linear} time, that is:
    \[
        \mathcal{O}(|V| + |E|).
    \]
\end{itemize}
