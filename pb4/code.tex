\subsection*{a)}

\textbf{Property:} A graph $G=(V,E)$ with $|G| \ge 3$ is $2$-connected if it has no articulation points and is connected (an articulation point is a vertex $a$ whose removal increases the number of connected components of $G$; $G-a$ is not connected).

\noindent\textbf{Proof:} For $G_1$ to be $2$-connected, the following must hold:
\begin{itemize}
    \item $|G_1|\ge 3$ (but we know $|G|\ge 3 \Rightarrow  |G_1|=|G|+1 \ge 4$)
    \item $G_1$ has no articulation points ($\forall a \in V(G_1), G_1-a$ is connected)
\end{itemize}

{\setlength{\leftmargini}{50pt}
\textit{\textbf{Case I:}} the removed vertex ($w$) is exactly $x$ (added in $G_1$)
\begin{itemize}
    \item $G_1-x=G$
    \item We know $G$ is $2$-connected $\Rightarrow G$ is connected. Since $G_1-x=G \Rightarrow G_1-x$ is connected.
\end{itemize}

\textit{\textbf{Case II:}} the removed vertex ($w$) belongs to the original graph ($G$)
\begin{itemize}
    \item $V(G_1)=(V(G)\setminus \{w\})\cup \{x\}$
    \item We know $G$ is $2$-connected $\Rightarrow G - w$ is connected 
    \item All vertices $V(G)\setminus \{w\}$ are connected to each other in $G-w$
    \item We show that $x$ is connected to this connected component
\end{itemize}

\textit{\textbf{Subcase IIa:}} $w \ne u$, $w \ne v$
\begin{itemize}
    \item In $G_1 - w$, $x$ is still connected to vertices $u,v$
    \item $u$ belongs to the connected component of $G-w$, and $x$ is connected to $u$, which is connected to all other vertices in $V(G) \setminus \{w\}$ \textit{(similarly, we could also use vertex $v$)} 
    \item $G_1-w$ is connected
\end{itemize}

\textit{\textbf{Subcase IIb:}} $w=u$
\begin{itemize}
    \item In $G_1-u$, $x$ loses the edge to $u$, but remains connected to $v$
    \item $v \ne u$, and $v$ belongs to the connected component of $G-u$
    \item We know $G-u$ is connected since $G$ is $2$-connected
    \item $x$ is connected to $v$, and $v$ is connected to all other vertices in $V(G)\setminus \{u\}$
    \item $x$ is connected to the entire connected component of $G-u$
    \item $G_1-u(u=w)$ is connected
\end{itemize}

\textit{\textbf{Subcase IIc:}} $w=v$ \textit{(analogous to Subcase IIb)\\\\}
\textbf{Conclusion:} From the two cases that cover all possibilities, we can conclude that $G_1$ is $2$-connected.
}

\subsection*{b)}
We want to prove an equivalence. We will prove each direction of the equivalence separately.

\begin{itemize}
\item \textbf{1. The $\Rightarrow$'' direction.} If $G$ is $2$-connected, then for any three distinct vertices $u, v$, and $w$ there exists a path from $u$ to $v$ that contains the vertex $w$. \begin{itemize}[label=$\circ$] \item We know that $G$ is $2$-connected. \item We construct a new graph $G'$ by adding a vertex $x$ and edges $(x,u)$ and $(x,v)$. \item According to the proof in point \textbf{a)}, $G'$ is $2$-connected. \item We choose the (distinct) vertices $x$ and $w \Rightarrow$ there exist two internally disjoint paths from $x$ to $w$ (they have no vertices in common except the endpoints). \item Let $D_1$ and $D_2$ be the two disjoint paths from $x$ to $w$ in $G'$. \item From the way we added the vertex $x$, it only has edges to $u$ and $v$. \item Since $D_1$ and $D_2$ are internally disjoint, they leave the vertex $x$ through different edges, as follows: \begin{itemize}[label=$\rightarrow$] \item $D_1: x\leadsto u\leadsto...\leadsto w$ \item $D_2: x\leadsto v\leadsto...\leadsto w$ \end{itemize} \item $D_{uw}$ (the path from $u$ to $w$) and $D_{vw}$ (the path from $v$ to $w$) share only the vertex $w$. \item Thus, we can construct the desired path: $u\rightarrow D_{uw}\rightarrow w \rightarrow D_{wv}\rightarrow v$ (where $D_{wv}$ is the reverse of the path $D_{vw}$). \end{itemize} \item \textbf{2. The $\Leftarrow$'' direction.} If for any three distinct vertices $u, v$, and $w$ there exists a path from $u$ to $v$ that contains the vertex $w$, then $G$ is $2$-connected ($G$ is connected and has no articulation points).
\begin{itemize} [label=$\circ$]
\item $|G| \ge 3$ (assumption)
\item We arbitrarily choose a vertex $a \in V(G)$ which we will remove.
\item To show that $a$ is not an articulation point, we will show that $G-a$ is connected $\Leftrightarrow \forall b,c \in V(G-a), (b\ne c)$ they are connected by a path in $G-a$.
\item Assumption: for any three distinct vertices $u, v$, and $w$ there exists a path from $u$ to $v$ that contains the vertex $w$.
\item Let: $u=a, v=b, w=c$
\item By the assumption, there must exist a path from $a$ to $b$ that contains $c$. That is, there exists a path $D: a\leadsto...\leadsto c\leadsto...\leadsto b$.
\item $D$ is a path $\Rightarrow$ all vertices on the path are distinct, and $b$ is the final vertex ($b$ appears in $D_{cb}$ but does not appear in $D_{ac}$).
\item $a \notin D_{cb} \Rightarrow D_{cb}$ is a valid path in $G-a$, and $b$ and $c$ were chosen arbitrarily $\Rightarrow G-a$ is connected.
\item The vertex $a$ was chosen arbitrarily $\Rightarrow G$ has no articulation points $\Rightarrow G$ is $2$-connected.
\end{itemize}
\end{itemize}
