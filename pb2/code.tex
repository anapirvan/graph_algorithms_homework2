\subsection*{a)}
Let $c':E\to \mathbb{R}$ be the new cost function.\
Assume that $T$ is no longer an MST with respect to $c' \Rightarrow$
there exists an MST in $G$, denoted $T'$, whose cost is smaller than that of $T$.\
Since only the edge $uv$ has its cost decreased ($c(uv)-\alpha(uv) \geq 0$, because the necessary condition for removing the edge $uv$ from $G$ is that its cost be at least equal to the maximum cost of the edges on the path from $u$ to $v$ that does not go through $uv$), and $T'$ has smaller cost than $T$, the only possibility is that $T'$ contains the edge $uv$.\
If we remove the edge $uv$ from $T'$, then the tree becomes disconnected and two connected components appear, which we will consider the sets of our cut.\
Then $uv$ must be the only minimum-cost edge connecting a node in $u$’s set with a node in $v$’s set. Otherwise, we would contradict the fact that $T'$ must necessarily contain the edge $uv$.\
But in $T$ there is an edge on the path from $u$ to $v$ that connects two nodes from different sets of the cut and whose cost is less than or equal to the cost of edge $uv$ ($c'(uv)=c(uv)-(c(uv)-\alpha(uv)-x)$, where $0\leq x\leq c(uv)-\alpha(uv) \Rightarrow c'(uv)=\alpha(uv)+x$, with $0\leq x\leq c(uv)-\alpha(uv) \Rightarrow c'(uv) \geq \alpha(uv)$) $\Rightarrow$ $uv$ is not the only minimum-cost edge $\Rightarrow$ the cost of $T'$ is greater than or equal to the cost of $T$ (contradiction) $\Rightarrow$ $T$ is an MST.

\subsection*{b)}

$\beta(uv)$ represents the minimum cost of the edges $xy \in E \setminus E'$, where $xy$ is an edge such that if we add it to $T$, then a cycle is formed that contains edge $uv$ (in $T$ we have the path from $x$ to $y$ that contains edge $uv$, but not edge $xy$).\
$c(uv)\leq \beta(uv)$, since we know that $T$ is an MST with respect to $c$, and edge $uv$ must have cost at most equal to the minimum cost of an edge $xy$ in order for the minimum spanning tree property to hold.\
Let $c':E\to\mathbb{R}$ be the new cost function.\
$c'(uv)=c(uv)+(\beta(uv)-c(uv)-x)$, where $0 \leq x \leq \beta(uv)-c(uv)$ (we add to the cost of the edge at most $\beta(uv)-c(uv)$) $\Rightarrow c'(uv)=\beta(uv)-x \Rightarrow c'(uv)\leq \beta(uv)$.\
Assume that $T$ is not an MST with respect to $c' \Rightarrow$ there exists $T'$, an MST with respect to $c'$, whose cost is smaller than that of $T$.
Since only the cost of edge $uv$ is increased, and $T'$ has a smaller cost than $T$, then $T'$ will not contain edge $uv$, but will contain one of the edges $xy$ that ensures the path from $u$ to $v$.\
If we remove edge $xy$ from $T'$, then the tree becomes disconnected and two connected components appear, $C1$ and $C2$, where $x \in C1$ and $y \in C2$. Then $xy$ should be the only minimum-cost edge that connects $C1$ and $C2$.\
We know that $c(xy)\geq \beta(uv)$ (from the definition of $\beta(uv)$) and that $c'(uv)\leq \beta(uv) \Rightarrow c'(uv) \leq c(xy)=c'(xy)$ ($c'$ differs from $c$ only for edge $uv$).\
Thus in $T$ there exists edge $uv$ on the path from $x$ to $y$, which belongs to the cut generated by $T'$, and which has cost less than or equal to the cost of edge $xy \Rightarrow$ the cost of $T'$ is greater than or equal to the cost of $T$ (contradiction) $\Rightarrow T$ is an MST.